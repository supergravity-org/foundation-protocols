\documentclass{article}
\usepackage[utf8]{inputenc}
\usepackage[german]{babel}

\usepackage{minutes}

\begin{document}

\begin{Minutes}{Supergravity Foundation}
 \subtitle{Vorstandstreffen 2016.}
 \moderation{Daniel Poelzleithner}
 \minutetaker{Daniel Poelzleithner}
 \participant{Iwona Brych, Daniel Poelzleithner}
 \missing[Daniel Plominski]
 \minutesdate{22.03.2016}
 \starttime{18:00}
 \endtime{21:00}
 \location{Wohnung von Iwona Brych}

\maketitle
\tableofcontents

\topic{Satzungsänderung}
\subtopic{Anpassen der Satzung bezüglich Mitgliedschaft}
Es wurde beschlossen einen Punkt 3 unter §4 Mitgliedschaft der Satzung einzufügen.
Folgender Wortlaut wurde erarbeitet der den Anmerkungen des Amtsgerichts folge leistet und mit dem Willen der beschlossen Satzung weitgehenst entspricht.
\begin{quote}
Der Beitritt erfolgt schriftlich oder über eine vom Verein bereitgestellten Webseite; bei Kindern und Jugendlichen bedarf sie der schriftlichen Zustimmung eines gesetzlichen Vertreters. Die Ablehnung eines Aufnahmeantrages kann nur durch den Vorstand, dessen Entscheidung keiner Begründung bedarf, erfolgen.
\end{quote}
\subtopic{Anpassen der Satzung bezüglich Einwilligung}
Es wurde beschlossen einen Punkt 4 unter §4 Mitgliedschaft einzufügen um sicherzustellen, dass das Mitglied mit der Satzung einverstanden ist.

\begin{quote}
Mit dem Beitritt erkennt das Mitglied die Satzung an.
\end{quote}

\task[done]{Daniel Polzleithner, Iwona Brych}[heute]{Es wurde wärend der Sitzung die notwendigen Änderung erarbeitet und beschlossen}
\task[pending]{Daniel Poelzleithner}{Brief verschicken}


\signature{Daniel Poelzleithner}

\signature{Iwona Brych}

\end{Minutes}
\end{document}
